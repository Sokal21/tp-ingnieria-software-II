\documentclass[%
  fleqn,colorlinks,linkcolor=blue,citecolor=blue,urlcolor=blue]{eptcs}
\usepackage[latin1]{inputenc}
\usepackage[spanish]{babel}
\usepackage{amsfonts}
\usepackage{amsmath}
\usepackage{amssymb}
\usepackage{amsthm}
\usepackage{z-eves}
\usepackage{framed}
\usepackage[latin1]{inputenc}
\usepackage[spanish]{babel}
\usepackage{xspace}

\newcommand{\desig}[2]{\item #1 $\approx #2$}
\newenvironment{designations}
  {\begin{leftbar}
    \begin{list}{}{\setlength{\labelsep}{0cm}
                   \setlength{\labelwidth}{0cm}
                   \setlength{\listparindent}{0cm}
                   \setlength{\rightmargin}{\leftmargin}}}
  {\end{list}\end{leftbar}}

\newcommand{\setlog}{$\{log\}$\xspace}

\def\titlerunning{\setlog}
\def\authorrunning{M. Cristi�}

\title{Trabajo Pr�ctico para Ingenier�a de Software}
\author{Maximiliano Cristi�}

\date{2018}

\begin{document}
\thispagestyle{empty}
\maketitle


\section{Requerimientos}
El sistema Agenda de Cumplea�os permite mantener un registro de los cumplea�os de los conocidos del usuario.

El sistema debe permitir asociar un nombre de persona con su fecha de cumplea�os.

Adem�s debe permitir encontrar la fecha de cumplea�os de una persona dada y el nombre de todas las personas que cumplen a�os en una fecha dada.


\section{Especificaci�n}
Comenzamos dando algunas designaciones.

\begin{designations}
\desig{$n$ es un nombre}{n \in NAME}
\desig{$d$ es una fecha}{d \in DATE}
\desig{$k$ es el nombre de una persona cuyo cumplea�os hay registrar}{k \in known}
\desig{La fecha de cumplea�os de la persona $k$}{birthday~k}
\end{designations}

Entonces introducimos los siguientes tipos b�sicos.

\begin{zed}
[NAME,DATE]
\end{zed}

Ahora podemos definir el espacio de estados de la especificaci�n de la siguiente forma.

\begin{schema}{BirthdayBook}
known: \power NAME \\
birthday: NAME \pfun DATE
\end{schema}

El estado inicial de la agenda de cumplea�os es el siguiente.

\begin{schema}{BirthdayBookInit}
BirthdayBook
\where
known = \emptyset \\
birthday = \emptyset
\end{schema}

El siguiente esquema describe los predicados que deber�an ser invariantes de estado.

\begin{schema}{BirthdayBookInv}
BirthdayBook
\where
known = \dom birthday
\end{schema}

La primera operaci�n que modelamos es c�mo agregar una fecha de cumplea�os a la agenda. Como siempre modelamos primero el caso exitoso, luego los errores y finalmente integramos todo en una �nica expresi�n de esquemas.

\begin{schema}{AddBirthdayOk}
\Delta BirthdayBook \\
name?:NAME \\
date?:DATE
\where
name? \notin known \\
known' = known \cup \{name?\} \\
birthday' = birthday \cup \{name? \mapsto date?\}
\end{schema}

\begin{schema}{NameAlreadyExists}
\Xi BirthdayBook \\
name?:NAME
\where
name? \in known
\end{schema}

\begin{zed}
AddBirthday == AddBirthdayOk \lor NameAlreadyExists
\end{zed}

La segunda operaci�n a especificar corresponde a mostrar el cumplea�os de una persona dada.

\begin{schema}{FindBirthdayOk}
\Xi BirthdayBook \\
name?:NAME \\
date!: DATE
\where
name? \in known \\
date! = birthday(name?)
\end{schema}

\begin{schema}{NotAFriend}
\Xi BirthdayBook \\
name?:NAME
\where
name? \notin known
\end{schema}

\begin{zed}
FindBirthday == FindBirthdayOk \lor NotAFriend
\end{zed}

Finalmente tenemos una operaci�n que nos lista los nombres de las personas cuya fecha de cumplea�os es hoy.

\begin{schema}{Remind}
\Xi BirthdayBook \\
today?:DATE \\
cards!: \power NAME
\where
cards! = \dom(birthday \rres \{today?\})
\end{schema}


\section{Simulaciones}

La primera simulaci�n es la siguiente:
\begin{verbatim}
birthdayBookInit(S0)             & addBirthday(S0,maxi,160367,S1) & 
addBirthday(S1,'Yo',201166,S2)   & findBirthday(S2,'Yo',C,S3) &
addBirthday(S3,'Otro',201166,S4) & remind(S4,160367,Card,S5) &
remind(S5,201166,Card1,S6).
\end{verbatim}
cuya primera respuesta es la siguiente:
\begin{verbatim}
S0 = {[known,{}],[birthday,{}]},  
S1 = {[known,{maxi}],[birthday,{[maxi,160367]}]},  
S2 = {[known,{maxi,Yo}],[birthday,{[maxi,160367],[Yo,201166]}]},  
C = 201166,  
S3 = {[known,{maxi,Yo}],[birthday,{[maxi,160367],[Yo,201166]}]},  
S4 = {[known,{maxi,Yo,Otro}],
      [birthday,{[maxi,160367],[Yo,201166],[Otro,201166]}]},  
Card = {maxi},  
S5 = {[known,{maxi,Yo,Otro}],
      [birthday,{[maxi,160367],[Yo,201166],[Otro,201166]}]},  
Card1 = {Yo,Otro},  
S6 = {[known,{maxi,Yo,Otro}],
      [birthday,{[maxi,160367],[Yo,201166],[Otro,201166]}]}
\end{verbatim}

La segunda simulaci�n es la siguiente:
\begin{verbatim}
S0 = {[known,{maxi,caro,cami,alvaro}],
      [birthday,{[maxi,160367],[caro,201166],[cami,290697],[alvaro,110400]}]} &
addBirthday(S0,'Yo',160367,S1) & remind(S1,160367,Card,S1).
\end{verbatim}
cuya primera respuesta es la siguiente:
\begin{verbatim}
S0 = {[known,{maxi,caro,cami,alvaro}],
      [birthday,{[maxi,160367],[caro,201166],[cami,290697],[alvaro,110400]}]},  
S1 = {[known,{maxi,caro,cami,alvaro,Yo}],
      [birthday,{[maxi,160367],[caro,201166],[cami,290697],[alvaro,110400],
                 [Yo,160367]}]},  
Card = {maxi,Yo}
\end{verbatim}


\section{Demostraciones con \setlog}

\paragraph{Primera demostraci�n con \setlog.}

Demuestro que $AddBirthday$ preserva el invariante $BirthdayBookInv$, o sea el siguiente teorema:
\begin{theorem}{AddBirthdayPI}
BirthdayBookInv \land AddBirthday \implies BirthdayBookInv'
\end{theorem}
el cual en \setlog se escribe de la siguiente forma:
\begin{verbatim}
S = {[known,K],[birthday,B]} &
S_ = {[known,K_],[birthday,B_]} &
dom(B,K) &
addBirthday(S,N,C,S_) &
ndom(B_,K_).
\end{verbatim}

\paragraph{Segunda demostraci�n con \setlog.}
Demuestro que $AddBirthday$ preserva el invariante $birthday \in NAME \pfun DATE$, o sea el teorema: 
\begin{theorem}{BirthdayIsPfun}
birthday \in NAME \pfun DATE \land AddBirthday \implies birthday' \in NAME \pfun DATE
\end{theorem}
el cual en \setlog se escribe de la siguiente forma:
\begin{verbatim}
S = {[known,K],[birthday,B]} &
S_ = {[known,K_],[birthday,B_]} &
dom(B,K) &
pfun(B) &
addBirthday(S,N,C,S_) &
npfun(B_).
\end{verbatim}
donde tuve que agregar como hip�tesis que el dominio de $birthday$ es igual a $known$.


\section{Demostraci�n con Z/EVES}

% read "/home/mcristia/fceia/is/material/setlog/ejemplo-tp.tex";

\begin{theorem}{AddBirthdayPI}
BirthdayBookInv \land AddBirthday \implies BirthdayBookInv'
\end{theorem}

\begin{zproof}[AddBirthdayPI]
invoke AddBirthday;
split AddBirthdayOk;
cases;
prove by reduce;
next;
prove by reduce;
next;
\end{zproof}

\section{Casos de prueba}
El script que us� para generar casos de prueba con Fastest es el siguiente:
\begin{verbatim}
loadspec fastest.tex        
selop AddBirthday
genalltt
addtactic AddBirthday_DNF_1 SP \cup birthday \cup \{name? \mapsto date?\}
genalltt
genalltca
\end{verbatim}
Es decir que gener� casos de prueba para la operaci�n $AddBirthday$ aplicando DNF y SP la expresi�n $birthday \cup \{name? \mapsto date?\}$ pero solo para particionar la clase de prueba $AddBirthday\_DNF\_1$.

De esta forma Fastest gener� casos de prueba para todas las clases satisfacibles. Los casos de prueba son los siguientes:

\begin{schema}{AddBirthday\_ SP\_ 2\_ TCASE}\\
 AddBirthday\_ SP\_ 2 
\where
 name? = NAMENameInput \\
 known =~\emptyset \\
 birthday =~\emptyset \\
 date? = DATEDateInput
\end{schema}


\begin{schema}{AddBirthday\_ SP\_ 4\_ TCASE}\\
 AddBirthday\_ SP\_ 4 
\where
 name? = NAMENameInput \\
 known =~\emptyset \\
 birthday =~\emptyset \\
 date? = DATEDateInput
\end{schema}


\begin{schema}{AddBirthday\_ SP\_ 6\_ TCASE}\\
 AddBirthday\_ SP\_ 6 
\where
 name? = NAMENameInput \\
 known =~\emptyset \\
 birthday = \{ ( NAMENameInput \mapsto DATEDateInput ) , ( NAME1014 \mapsto DATE1020 ) \} \\
 date? = DATEDateInput
\end{schema}


\begin{schema}{AddBirthday\_ SP\_ 7\_ TCASE}\\
 AddBirthday\_ SP\_ 7 
\where
 name? = NAMENameInput \\
 known =~\emptyset \\
 birthday = \{ ( NAMENameInput \mapsto DATEDateInput ) \} \\
 date? = DATEDateInput
\end{schema}


\begin{schema}{AddBirthday\_ DNF\_ 2\_ TCASE}\\
 AddBirthday\_ DNF\_ 2 
\where
 name? = NAMENameInput \\
 known = \{ NAMENameInput \} \\
 birthday =~\emptyset \\
 date? = DATEDateInput
\end{schema}

\end{document}










